\documentclass[a4paper,12pt,ngerman,oneside]{article}

\usepackage{color}
\usepackage[utf8]{inputenc}
\usepackage[german]{babel}
\usepackage{fancyhdr}
\usepackage{amssymb}
\usepackage{url}
\usepackage{amsmath}
\usepackage[document]{ragged2e}
\usepackage{framed}

%Dokument Metadaten
\title{Computermathematik 1 - Übung 1 - Aufgabe 1}
\date{10. Oktober 2017}
\author{cernoel\\ }

\begin{document} 
  %Definiere Seiten-Stil für folgende Seiten (Header, Footer,..)
  \pagestyle{fancy}
  \fancyhf{}
  \rhead{cernoel\LaTeX}
  \lhead{Computermathematik - Übung1 - Aufgabe 1}
  \rfoot{Seite \thepage}
  
  %Anfang des Artikels
  \textbf{Auszug} aus Mitschrift Analysis T1 vom 06.10.17 \break
  in Bezug auf das Analysis T1 Wintersemester 2017/18 - Buch (Institut für Mathematik A von 
  Peter Grabner/Christian Elsholz) und Wikipedia-Eintrag: \url{https://de.wikipedia.org/wiki/Peano-Axiome} \break
  zu Punkt 1.1 Prinzip der vollständigen Induktion \break
  \break
  \textbf{Peano Axiome} wurden 1889 vom italienischen Mathematiker Giuseppe Peano formuliert und sind 
  f\"unf Axiome (Aussageform die immer als wahr angenommen werden, also beweislos vorausgesetzt werden / sind das 
  Gegenteil von Thesen) die nat\"urliche Zahlen $\mathbb{N}$ und ihre Eigenschaften beschreiben:
  
  %Aufzaehlung der Axiome
  \begin{framed}
    \begin{enumerate}
      \item $0 \in \mathbb{N}_0$
      \begin{enumerate}
        \item 0 ist eine nat\"urliche Zahl
      \end{enumerate}

      \item $\forall n \in \mathbb{N}_0:S(n) \in \mathbb{N}_0$
      \begin{enumerate}
        \item Jede nat\"urliche Zahl \textit{n} hat \textit{(n+1)} als Nachfolger.
        \item $S(n) = (n+1)$
        \item lt. Wikipedia auch: $\forall n(n \in \mathbb{N}_0 \Rightarrow n'\in \mathbb{N}_0 $) wobei $n' = (n + 1)$
      \end{enumerate}

      \item $\forall n \in \mathbb{N}_0:S(n) \neq 0$
      \begin{enumerate}
        \item 0 ist kein Nachfolger einer nat\"urlichen Zahl
        \item Wiki: $\forall n(n\in \mathbb{N}_0 \Rightarrow n' \neq 0)$
      \end{enumerate}

      \item $\forall m,n \in \mathbb{N}_0:S(m)=S(n) \Rightarrow m=n$
      \begin{enumerate}
        \item Nat\"urliche Zahlen mit gleichem Nachfolger sind gleich
        \item Wiki: $\forall n,m(m,n \in \mathbb{N}_0 \Rightarrow (m' = n' \Rightarrow m = n)) $
      \end{enumerate}

      \item Jede Menge x mit den Eigenschaften oben enth\"allt $\mathbb{N}_0$
    \end{enumerate}
  \end{framed}
  Bemerkung: Jede nicht leere Teilmenge der nat\"urlichen Zahlen besitzt ein \textbf{kleinstes} Element = Wohlordnung der nat\"urlichen
  Zahlen 
\end{document}