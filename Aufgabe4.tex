\documentclass[a4paper,12pt,ngerman,oneside]{article}

\usepackage[utf8]{inputenc} %now we can use mutations (Ö, Ä,..)
\usepackage[german]{babel}
\usepackage{charter} %activate Font Family Charter for whole document
\usepackage[linesnumbered,lined,boxed,commentsnumbered]{algorithm2e} %for pseudo code
\usepackage{titling} %used for maketitle
\usepackage{listings, verbatim} %used for real code in document
\usepackage{caption} %captions in source - code examples
\usepackage{xcolor} %so we can use Color-Names (gray,...)
\usepackage{amsmath, amssymb} %math symbols and stuff

%Listings-Caption Boxes Settings
\DeclareCaptionFont{white}{\color{white}}
\DeclareCaptionFormat{listing}{%
  \parbox{\textwidth}{\colorbox{gray}{\parbox{\textwidth}{#1#2#3}}\vskip-4pt}}
\captionsetup[lstlisting] %label settings for listings
{
	format=listing,
	labelfont=white,
	textfont=white
}
\lstset % lstlisting settings
{
	frame=lrb, %frame left right bottom
	xleftmargin=\fboxsep, %box to text margin left
	xrightmargin=-\fboxsep, %box to text margin right
	keywordstyle=\bfseries,
	language=C,
	numbers=left,
	breaklines=false,
	tabsize=4,
}
%listing style found: https://tex.stackexchange.com/questions/14967/source-code-listing-with-frame-around-code


%Dokument Metadaten
\title{Computermathematik 1 - Übung 2 - Aufgabe 4}
\date{\today}
\author{cernoel}

\begin{document} 
  \maketitle
  \begin{abstract}
    Man implementiere einen Algorithmus zum naiven Testen, ob eine Eingabezahl
    eine Primzahl ist oder nicht. Darstellung des Algorithmus mittels Pseudocode 
    und C-Implementation.
  \end{abstract}
  \newpage
  \tableofcontents 
  
  \newpage
  \section{Einleitung}
  Man implementiere einen Algorithmus zum naiven Testen, ob eine Eingabezahl 
  $N \in \mathbb{N}$ eine Primzahl ist oder nicht.(Zur Erinnerung: eine Zahl 
  $N \in \mathbb{N}$ ist eine Primzahl, wenn sie nur durch 1 und sich selbst 
  teilbar ist.) Eine sehr einfache Testmöglichkeit ist, dass man jede Zahl $m \in 
  \{ 2, 3, 4, ..., N - 2, N - 1 \}$ testet, ob sie die Zahl $N$ teilt. Findet man 
  keine solche Zahl $m$, so ist $N$ offensichtlich eine Primzahl.
  
  \newpage
  \section{Codebeispiele}
  \subsection{Pseudo-Implementation}
  %for detailed info look: https://www.tug.org/texlive/Contents/live/texmf-dist/doc/latex/algorithm2e/algorithm2e.pdf
  \begin{algorithm}[H] %H forces the algorithm to stay in place
    \SetAlgoLined %draws lines between function start and end
    \KwData{Input as Integer} %description input data
    \KwResult{to prime or not to prime, that is the question} %Result

    \While{counter \textless \ input}{
      result = divide input with counter\;
    \eIf{result is greater than 0}{
      we found our divider and input is no prime;
      }{
      move on with the loop;
      }
      when we have not found a divider;
      then the input is a prime;
    }
  \caption{Simple Primefinder Algorithm}
  \end{algorithm}
  
  \newpage
  \subsection{C-Implementation}
  
  \belowcaptionskip=-10pt
  \begin{lstlisting}[label=example1,caption=Primefinder in C with 997 as Testnumber, basicstyle=\small]
#include "stdio.h"

// Function isPrime(unsigned long int input)
// returns 0 if prime, or number > 0, where number is 
// the divider.
int isPrime(unsigned long int input){
int isPrime = 1;

for(int x = 2; x < input; x++)
{
  if (input % x == 0)
  {
	isPrime = 0;
	break; //loop
  }
  
  isPrime = x; //sets return to Divider
  }
  
  return isPrime;
}

int main()
{
  unsigned long int input = 997; //input variable
  int isInputPrime = isPrime(input);

  if(isInputPrime == 0){
	printf("%ld is a Prime!\n", input);
  } else {
	printf("No Prime! First Divider: %d\n", isInputPrime);
  }
}
  \end{lstlisting}
  

  
  Donald E.~Knuth  showed~\cite{artof}...
  \bibliographystyle{plain}
  
  \newpage
  \section{Literatur}
  \bibliography{literatur}

\end{document}