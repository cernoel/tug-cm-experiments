\documentclass[a4paper,12pt,ngerman,oneside]{article}

\usepackage[ngerman]{babel} %new German babel .. used for word-wrap,..

\usepackage[utf8]{inputenc} %Für Texte mit Sonderzeiche und Umlaute
\usepackage[ngerman]{babel} %new German babel .. used for word-wrap,..
\usepackage{fancyhdr} %for fancy header/footer manipulation
\usepackage[table]{xcolor} %coloring the table
\usepackage{charter} %activate Font Family Charter for whole document

\usepackage{tikz} %TikZ drawing package itself
% To use the package tkz-euclide, place the following lines in the preamble of your LaTeX document.
% https://tex.stackexchange.com/questions/42611/list-of-available-tikz-libraries-with-a-short-introduction
\usepackage{tkz-euclide} %to use geometric calculation stuff in tikz
\usetkzobj{all} 

\usetikzlibrary{through} %for through calculations
\usetikzlibrary{intersections} %for intersect calculations


\begin{document}
%Definiere Seiten-Stil für folgende Seiten (Header, Footer,..)
\pagestyle{fancy} %activates the Style for this document
\fancyhf{} % clear header/footer
\rhead{cernoel\LaTeX} %on header right side
\lhead{Computermathematik 1 - Übung 3 - Aufgabe 6} %on header left side
\rfoot{Seite \thepage} %on footer right side
\centering %center everything

\begin{flushleft} %text align left
	1.) Mittelpunkt / Schwerpunkt bei einem Dreieck \break
	\break
	Um auf den Schwerpunkt (M) eines Dreiecks zu kommen, nimmt man
	von allen Seiten den Schnittpunkt der Seitenhalbierenden, was heißt, die halbe
	Länge der Seite. zB.: die Seitenhalbierende AB wird berechnet indem man die Länge
	der Strecke zwischen Punkt A und B durch 2 rechnet. Man zieht dann von dieser Halbierung
	eine Linie zum gegenüberliegenden Punkt (C). Das macht man mit allen Seiten. Logischerweise
	treffen sich alle Linien dann genau im Schwerpunkt.
	\break
\end{flushleft}

%Triangle
\begin{tikzpicture}

	\coordinate [label=left:$A$] (A) at (2,4);
	\coordinate [label=left:$B$] (B) at (4,8);
	\coordinate [label=right:$C$] (C) at (8,2);
	
	\draw (A) -- (B) [name path=AB] coordinate (AB);
	\draw (B) -- (C) [name path=BC] coordinate (BC);
	\draw (C) -- (A) [name path=CA] coordinate (CA);
	
	% Seitenhalbierende AB
	\coordinate [label=left:$AB/2$] (SHAB) at ($(A)!0.5!(B)$);
	\draw[dotted,red, name path=SHABC] (SHAB) -- (C);
	
	% Seitenhalbierende BC
	\coordinate [label=right:$BC/2$](SHBC) at ($(B)!0.5!(C)$);
	\draw[dotted,red, name path=SHBCA] (SHBC) -- (A);  
	
	% Seitenhalbierende CA
	\coordinate [label=left:$CA/2$](SHCA) at ($(C)!0.5!(A)$);
	\draw[dotted,red, name path=SHBAC] (SHCA) -- (B);
	
	% Mittelpunkt IM (intersect von zwei Seitenhalbierenden)
	\draw circle [name intersections={of=SHBCA and SHBAC, by=IM}];
	\node [fill=red,inner sep=2pt,label=-90:$M$] at (IM) {};

\end{tikzpicture}
\break

\newpage
\begin{flushleft} %text align left
	2.) Endlicher Automat: Division / 2
	\break
	\break
	Dieser Automat beschreibt den Rechts-Shift Vorgang bei einer
	ein-Band-Turing-Maschine. Wenn am Anfang eine Null gelesen wird, wird
	weitergelesen bis die erste Eins kommt. Die Maschine wechselt in den Shift
	Status und sucht nach der nächsten Null. Sollte keine Null auffindbar sein, 
	ist keine Division möglich, da die Zahl ungerade ist. Wenn eine Null gefunden wird, 
	wechselt die Maschine wieder in den Start Status und sucht nach der nächsten Eins.
	Wenn keine eins mehr auffindbar ist, dann ist der Shift Vorgang beendet und die Division vollzogen.
	% Strukturdiagramm mit Pfeilen
	\begin{tikzpicture}
	\node (Start) at (2,8) [circle,shade,draw,minimum size=3cm] {Start};
	\node (Shift) at (8,8) [circle,shade,draw,minimum size=3cm] {Shift};
	
	\node (End) at (2,2) [circle,shade,draw,minimum size=3cm] {End};
	\node (notDiv) at (8,2) [circle,shade,draw,minimum size=3cm] {notDiv};
	
	\draw[->, blue!50, line width=1mm] (Start) to[bend left=320]
	  node[left] {underline} (End);
	  
	\draw[->, blue!50, line width=1mm] (Start) to[bend left=45]
	  node[above] {one} (Shift);
	
	\draw[->, blue!50, line width=1mm, looseness=8] (Start) to[out=140, in=160] node[left] {zero} (Start);
	
	\draw[->, blue!50, line width=1mm] (Shift) to[bend left=45]
	node[above] {zero} (Start);
	
	\draw[->, blue!50, line width=1mm, looseness=8] (Shift) to[out=40, in=60] node[right] {one} (Shift);
	
	\draw[->, blue!50, line width=1mm] (Shift) to[out=320, in=45] node[right] {underline} (notDiv);

 	\end{tikzpicture}
 	
 	
	
\end{flushleft}
  


\end{document}
