\documentclass[a4paper, 11pt, ngerman, twocolumn]{article}

\usepackage[a4paper, margin=2.5cm]{geometry} % für Randabstände/Bleeding/Margin
% https://www.sharelatex.com/learn/Page_size_and_margins

\usepackage[utf8]{inputenc} %Für Texte mit Sonderzeiche und Umlaute
\usepackage[ngerman]{babel} %New German babel .. für Wort-Umbrüche usw.
\usepackage{fancyhdr} %for fancy header/footer manipulation
\usepackage{tikz} %TikZ Zeichen Package
\usetikzlibrary{patterns} % um Muster in Flächen zu zeichnen
\usetikzlibrary{calc} %für Berechnungen innerhalb von Zeichnungen
\usetikzlibrary{intersections} %um mit Pfaden zu arbeiten
\usepackage{framed} %um einfach Boxen um etwas zu zeichnen
\usepackage{amsmath} %um Mathe-Formeln darzustellen.
\usepackage{booktabs} %Horizontale Linien

\setlength{\headheight}{14pt}

\newcommand{\myLineWidth}{0.5mm}

\begin{document}
	
	%Kopfzeile
	\pagestyle{fancy} %Aktiviert den Fancy-Stil-Modus
	\rhead{00000000} %Kopfzeile rechts
	\chead{Hausaufgabe 1} %Kopfzeile mitte 
	\lhead{cernoel} %Kopfzeile links
	
	\section{Aufgabenstellung}
	
	\begin{flushleft}
		Bestimme anhand der folgenden Skizze die gefragten Unbekannten.
		(Lösung zur Probe ist $15cm^2$) \break
		Die Flächenstücke $A$,$B$,$C$,$D$ haben den gleichen Umfang. $D$ = ? \break
	\end{flushleft}
	
	\begin{center}
	\begin{tikzpicture}[scale=0.55]
		
	% ==========
	% Rechteck A
	% ==========			
	% Koordinaten des Rechtecks bestimmen
	\coordinate (AA) at (0,1);
	\coordinate (AB) at (7,1);
	\coordinate (AC) at (7,0);
	\coordinate (AD) at (0,0);
	
	% Mittelpunkt des Rechtecks finden...
	% -----------------------------------
	% Seitenhalbierende (oben)
	\coordinate (A-SH-AA-AB) at ($(AA)!0.5!(AB)$);
	% Seitenhalbierende (unten)
	\coordinate (A-SH-AC-AD) at ($(AC)!0.5!(AD)$);
	% unsichtbaren Pfad von oben nach unten berechnen
	\path[name path=AHORIZ] (A-SH-AA-AB) -- (A-SH-AC-AD);
	% Seitenhalbierende (links)
	\coordinate (A-SH-AA-AD) at ($(AA)!0.5!(AD)$);
	% Seitenhalbierende (rechts)
	\coordinate (A-SH-AB-AC) at ($(AB)!0.5!(AC)$);
	% unsichtbaren Pfad von links nach rechts berechne 
	\path[name path=AVERT] (A-SH-AA-AD) -- (A-SH-AB-AC);
	% Schnittpunkt von Horizontale und Vertikale
	\path [name intersections={of=AHORIZ and AVERT, by=A-Mitte}];
	% ----------------------------------------
	
	% Zeichne die Box			
	\draw [line width=\myLineWidth] (AA)--(AB)--(AC)--(AD)--(AA);
	
	% Schreibe A am Schnittpunkt (in das Rechteck)
	\node at (A-Mitte) {\LARGE $A$};
	
	% ==========
	% Rechteck B
	% ==========
	% Koordinaten des Rechtecks bestimmen
	\coordinate (BA) at (0,0);
	\coordinate (BB) at (2,0);
	\coordinate (BC) at (2,-6);
	\coordinate (BD) at (0,-6);
	
	% Mittelpunkt des Rechtecks finden...
	% -----------------------------------
	% Seitenhalbierende (oben)
	\coordinate (A-SH-BA-BB) at ($(BA)!0.5!(BB)$);
	% Seitenhalbierende (unten)
	\coordinate (A-SH-BC-BD) at ($(BC)!0.5!(BD)$);
	% unsichtbaren Pfad von oben nach unten berechnen
	\path[name path=BHORIZ] (A-SH-BA-BB) -- (A-SH-BC-BD);
	% Seitenhalbierende (links)
	\coordinate (A-SH-BA-BD) at ($(BA)!0.5!(BD)$);
	% Seitenhalbierende (rechts)
	\coordinate (A-SH-BB-BC) at ($(BB)!0.5!(BC)$);
	% unsichtbaren Pfad von links nach rechts berechnen
	\path[name path=BVERT] (A-SH-BA-BD) -- (A-SH-BB-BC);
	% Schnittpunkt von Horizontale und Vertikale
	\path [name intersections={of=BHORIZ and BVERT, by=B-Mitte}];
	% ----------------------------------------
	
	% Zeichne die Box			
	\draw [line width=\myLineWidth] (BA)--(BB)--(BC)--(BD)--(BA);
	
	 % Schreibe B am Schnittpunkt (in das Rechteck)
	\node at (B-Mitte) {\LARGE $B$};
	
	
	% ==========
	% Rechteck C
	% ==========
	% Koordinaten des Rechtecks bestimmen			
	\coordinate (CA) at (2,0);
	\coordinate (CB) at (7,0);
	\coordinate (CC) at (7,-3);
	\coordinate (CD) at (2,-3);
	
	% Mittelpunkt des Rechtecks finden...
	% -----------------------------------
	% Seitenhalbierende (oben)
	\coordinate (A-SH-CA-CB) at ($(CA)!0.5!(CB)$);
	% Seitenhalbierende (unten)
	\coordinate (A-SH-CC-CD) at ($(CC)!0.5!(CD)$);
	% unsichtbaren Pfad von oben nach unten berechnen
	\path[name path=CHORIZ] (A-SH-CA-CB) -- (A-SH-CC-CD);
	% Seitenhalbierende (links)
	\coordinate (A-SH-CA-CD) at ($(CA)!0.5!(CD)$);
	% Seitenhalbierende (rechts)
	\coordinate (A-SH-CB-CC) at ($(CB)!0.5!(CC)$);
	% unsichtbaren Pfad von links nach rechts berechnen
	\path[name path=CVERT] (A-SH-CA-CD) -- (A-SH-CB-CC);
	% Schnittpunkt von Horizontale und Vertikale
	\path [name intersections={of=CHORIZ and CVERT, by=C-Mitte}];
	% ----------------------------------------
	
	% Zeichne die Box			
	\draw [line width=\myLineWidth] (CA)--(CB)--(CC)--(CD)--(CA);
	
	% Schreibe B am Schnittpunkt (in das Rechteck)
	\node at (C-Mitte) {\LARGE $C$};
	
		
	% ==========
	% Rechteck C
	% ==========
	% Koordinaten des Rechtecks bestimmen;
	\coordinate (DA) at (2,-3);
	\coordinate (DB) at (7,-3);
	\coordinate (DC) at (7,-6);
	\coordinate (DD) at (2,-6);
	
	% Mittelpunkt des Rechtecks finden...
	% -----------------------------------
	% Seitenhalbierende (oben)
	\coordinate (A-SH-DA-DB) at ($(DA)!0.5!(DB)$);
	% Seitenhalbierende (unten)
	\coordinate (A-SH-DC-DD) at ($(DC)!0.5!(DD)$);
	% unsichtbaren Pfad von oben nach unten berechnen
	\path[name path=DHORIZ] (A-SH-DA-DB) -- (A-SH-DC-DD);
	% Seitenhalbierende (links)
	\coordinate (A-SH-DA-DD) at ($(DA)!0.5!(DD)$);
	% Seitenhalbierende (rechts)
	\coordinate (A-SH-DB-DC) at ($(DB)!0.5!(DC)$);
	% unsichtbaren Pfad von links nach rechts berechnen
	\path[name path=DVERT] (A-SH-DA-DD) -- (A-SH-DB-DC);
	% Schnittpunkt von Horizontale und Vertikale
	\path [name intersections={of=DHORIZ and DVERT, by=D-Mitte}];
	% ----------------------------------------
	
	% Zeichne die Box			
	\draw [line width=\myLineWidth, pattern=dots] (DA)--(DB)--(DC)--(DD)--(DA);
	
	% Schreibe B am Schnittpunkt (in das Rechteck)
	\node at (D-Mitte) {\LARGE $D$};
	
	% Text auf Pfad AA -- AB
	\draw (AA) -- (AB) node [midway, above, sloped] (TextNode) {\LARGE $7cm$};
	
	% Text auf Pfad BD -- AA
	\draw (BD) -- (AA) node [midway, above, sloped] (TextNode) {\LARGE $7cm$};
		
	\end{tikzpicture}
	\end{center}
	
	\section{Bestimmung der Seiten}
	\subsection{Überlegungen}
		\begin{flushleft}
			Angenommen: Flächen $C$ und $D$ sind exakt gleich, haben also dieselbe
			Höhe und Breite, die gegenüberliegenden Seiten der Flächen
			haben die exakt gleiche Länge, sind also Rechtecke. \break
			\break
			Umfang U: $U_A = U_B = U_C = U_D$ \break
			Flächenformel bei Rechtecken: $U = 2 \cdot (a + b )$ \break
			Versuch die Flächenformeln gleichzusetzen und $x$,$y$ dadurch zu gewinnen \break
		\end{flushleft}

		\begin{center}	
		\begin{tikzpicture}[scale=0.6]
		% ==========
		% Rechteck A
		% ==========			
		% Koordinaten des Rechtecks bestimmen
		\coordinate (AA) at (0,1);
		\coordinate (AB) at (7,1);
		\coordinate (AC) at (7,0);
		\coordinate (AD) at (0,0);
		
		% Mittelpunkt des Rechtecks finden...
		% -----------------------------------
		% Seitenhalbierende (oben)
		\coordinate (A-SH-AA-AB) at ($(AA)!0.5!(AB)$);
		% Seitenhalbierende (unten)
		\coordinate (A-SH-AC-AD) at ($(AC)!0.5!(AD)$);
		% unsichtbaren Pfad von oben nach unten berechnen
		\path[name path=AHORIZ] (A-SH-AA-AB) -- (A-SH-AC-AD);
		% Seitenhalbierende (links)
		\coordinate (A-SH-AA-AD) at ($(AA)!0.5!(AD)$);
		% Seitenhalbierende (rechts)
		\coordinate (A-SH-AB-AC) at ($(AB)!0.5!(AC)$);
		% unsichtbaren Pfad von links nach rechts berechne 
		\path[name path=AVERT] (A-SH-AA-AD) -- (A-SH-AB-AC);
		% Schnittpunkt von Horizontale und Vertikale
		\path [name intersections={of=AHORIZ and AVERT, by=A-Mitte}];
		% ----------------------------------------
		
		% Zeichne die Box			
		\draw [line width=\myLineWidth] (AA)--(AB)--(AC)--(AD)--(AA);
		
		% Schreibe A am Schnittpunkt (in das Rechteck)
		\node at (A-Mitte) {\LARGE $A$};
		
		% ==========
		% Rechteck B
		% ==========
		% Koordinaten des Rechtecks bestimmen
		\coordinate (BA) at (0,0);
		\coordinate (BB) at (2,0);
		\coordinate (BC) at (2,-6);
		\coordinate (BD) at (0,-6);
		
		% Mittelpunkt des Rechtecks finden...
		% -----------------------------------
		% Seitenhalbierende (oben)
		\coordinate (A-SH-BA-BB) at ($(BA)!0.5!(BB)$);
		% Seitenhalbierende (unten)
		\coordinate (A-SH-BC-BD) at ($(BC)!0.5!(BD)$);
		% unsichtbaren Pfad von oben nach unten berechnen
		\path[name path=BHORIZ] (A-SH-BA-BB) -- (A-SH-BC-BD);
		% Seitenhalbierende (links)
		\coordinate (A-SH-BA-BD) at ($(BA)!0.5!(BD)$);
		% Seitenhalbierende (rechts)
		\coordinate (A-SH-BB-BC) at ($(BB)!0.5!(BC)$);
		% unsichtbaren Pfad von links nach rechts berechnen
		\path[name path=BVERT] (A-SH-BA-BD) -- (A-SH-BB-BC);
		% Schnittpunkt von Horizontale und Vertikale
		\path [name intersections={of=BHORIZ and BVERT, by=B-Mitte}];
		% ----------------------------------------
		
		% Zeichne die Box			
		\draw [line width=\myLineWidth] (BA)--(BB)--(BC)--(BD)--(BA);
		
		% Schreibe B am Schnittpunkt (in das Rechteck)
		\node at (B-Mitte) {\LARGE $B$};
		
		
		% ==========
		% Rechteck C
		% ==========
		% Koordinaten des Rechtecks bestimmen			
		\coordinate (CA) at (2,0);
		\coordinate (CB) at (7,0);
		\coordinate (CC) at (7,-3);
		\coordinate (CD) at (2,-3);
		
		% Mittelpunkt des Rechtecks finden...
		% -----------------------------------
		% Seitenhalbierende (oben)
		\coordinate (A-SH-CA-CB) at ($(CA)!0.5!(CB)$);
		% Seitenhalbierende (unten)
		\coordinate (A-SH-CC-CD) at ($(CC)!0.5!(CD)$);
		% unsichtbaren Pfad von oben nach unten berechnen
		\path[name path=CHORIZ] (A-SH-CA-CB) -- (A-SH-CC-CD);
		% Seitenhalbierende (links)
		\coordinate (A-SH-CA-CD) at ($(CA)!0.5!(CD)$);
		% Seitenhalbierende (rechts)
		\coordinate (A-SH-CB-CC) at ($(CB)!0.5!(CC)$);
		% unsichtbaren Pfad von links nach rechts berechnen
		\path[name path=CVERT] (A-SH-CA-CD) -- (A-SH-CB-CC);
		% Schnittpunkt von Horizontale und Vertikale
		\path [name intersections={of=CHORIZ and CVERT, by=C-Mitte}];
		% ----------------------------------------
		
		% Zeichne die Box			
		\draw [line width=\myLineWidth] (CA)--(CB)--(CC)--(CD)--(CA);
		
		% Schreibe B am Schnittpunkt (in das Rechteck)
		\node at (C-Mitte) {\LARGE $C$};
		
		
		% ==========
		% Rechteck C
		% ==========
		% Koordinaten des Rechtecks bestimmen;
		\coordinate (DA) at (2,-3);
		\coordinate (DB) at (7,-3);
		\coordinate (DC) at (7,-6);
		\coordinate (DD) at (2,-6);
		
		% Mittelpunkt des Rechtecks finden...
		% -----------------------------------
		% Seitenhalbierende (oben)
		\coordinate (A-SH-DA-DB) at ($(DA)!0.5!(DB)$);
		% Seitenhalbierende (unten)
		\coordinate (A-SH-DC-DD) at ($(DC)!0.5!(DD)$);
		% unsichtbaren Pfad von oben nach unten berechnen
		\path[name path=DHORIZ] (A-SH-DA-DB) -- (A-SH-DC-DD);
		% Seitenhalbierende (links)
		\coordinate (A-SH-DA-DD) at ($(DA)!0.5!(DD)$);
		% Seitenhalbierende (rechts)
		\coordinate (A-SH-DB-DC) at ($(DB)!0.5!(DC)$);
		% unsichtbaren Pfad von links nach rechts berechnen
		\path[name path=DVERT] (A-SH-DA-DD) -- (A-SH-DB-DC);
		% Schnittpunkt von Horizontale und Vertikale
		\path [name intersections={of=DHORIZ and DVERT, by=D-Mitte}];
		% ----------------------------------------
		
		% Zeichne die Box			
		\draw [line width=\myLineWidth, pattern=dots] (DA)--(DB)--(DC)--(DD)--(DA);
		
		% Schreibe B am Schnittpunkt (in das Rechteck)
		\node at (D-Mitte) {\LARGE $D$};
		
		% Text auf Pfad
		\draw (AA) -- (AB) node [midway, above, sloped] (TextNode) {$7cm$}; %Text rotieren.
		\draw (AB) -- (AC) node [midway, above, sloped] (TextNode) {\rotatebox{270}{\small $x$}}; %Text rotieren.
		\draw (BA) -- (BB) node [midway, below, sloped] (TextNode) {\tiny $7cm-y$};
		\draw (CA) -- (CB) node [midway, below, sloped] (TextNode) {$y$};
		\draw (BA) -- (BD) node [midway, below, sloped] (TextNode) {\rotatebox{180}{$7cm$}};
		\draw (CB) -- (CC) node [midway, above, sloped] (TextNode) {\rotatebox{90}{\small $z = (7cm-x)/2$}};
		\end{tikzpicture}
		\end{center}
		
	   \subsection{Umfangformeln}
	    
		\begin{equation*}
		\begin{array}{rl}
			U_A &=\ 2 \cdot (7cm + x) \\ 
			U_B &=\ 2 \cdot [(7cm - x) + (7cm-y)] \\
			U_C &=\ 2 \cdot [(\frac{7cm - x}{2}) + y] \\
			U_D &=\ U_C
		\end{array}
		\end{equation*}
		
		\subsection{Umfang gleichsetzen und Gleichung lösen}
		\begin{equation*}
		\begin{array}{rl}
			U_A &=\ U_B \\
			2 \cdot (7cm + x) &=\ 2 \cdot (7cm - x + 7cm - y)\\
			14cm + 2x &=\ 2 \cdot (7cm - x + 7cm - y)\\ 
			14cm + 2x &=\ 2 \cdot (14cm - x - y) \\
			14cm + 2x &=\ 28cm - 2x - 2y \quad \text{\textbar} +2x\\
			14cm + 4x &=\ 28cm - 2y \quad \text{\textbar} :2\\
			7cm + 2x &=\ 14cm - y \quad \text{\textbar} -7cm \quad \text{\textbar} +y \\
			I: 2x + y &=\ 7cm\\
			\\
			U_A &=\ U_C \\
			2 \cdot (7cm + x) &=\ 2 \cdot [(\frac{7cm - x}{2}) + y] \\
			14cm + 2x &=\ 7cm - x + 2y\\
			II: 3x - 2y &=\ -7cm \\	
			\end{array}
			\end{equation*}
		
		%Tabelle, getrennt bei &, lininen mit midrule
		\begin{equation*}
		\begin{array}{rl} 
		I: 2x + y &=\ 7cm \qquad \text{\textbar} \cdot2\\
		II: 3x - 2y &=\ -7cm \\
		\midrule
		I: 4x + 2y &=\ 14cm \\
		+ II: 3x - 2y &=\ -7cm \\
		\midrule
		7x &=\ 7cm \\
		x	&=\ 1cm \\ 
		\midrule
		\midrule
		I: 2 \cdot 1cm + y &=\ 7cm \qquad \text{\textbar} -2cm\\
		y &=\ 5cm \\
		\midrule
		\midrule
		\end{array}
		\end{equation*}
		
		\subsection{Einsetzen der Ergebnisse} 
	 	\begin{equation*}
	 	\begin{array}{rl}
		 	A &=\ 7cm \cdot x \\
		 	A &=\ 7cm \cdot 1cm \\
		 	  &=\ 7cm^2 \\
		 	B &=\ (7cm - y) \cdot (7cm - x) \\
		 	B &=\ (7cm - 5cm) \cdot (7cm - 1cm) \\
		 	  &=\  2cm \cdot 6cm \\
		 	  &=\ 12cm^2 \\
		 	C &=\ y \cdot z \\
		 	z &=\ 7cm - 1cm / 2 = 3cm \\
		 	C &=\ 5cm \cdot 3cm \\
		 	  &=\ 15cm^2 \\
		 	D &=\ C \\
		 	  &=\ 15cm^2 \\ 
	 	\end{array}
	 	\end{equation*}
	 	
		 \begin{center}
		 	\begin{framed}
 		 	Die gesuchte Fläche $D$ ergibt $15cm^2$.
 		 	\end{framed}
		 \end{center}
		 
 
		
\end{document}